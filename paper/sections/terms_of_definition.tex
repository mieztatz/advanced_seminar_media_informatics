\section{Terms of Definition}
\label{sec:terms_of_definition}
In this section the term \textit{intimate} is defined. Due to this it is considered what data is perceived as intimate and in what circumstances.

The question can not be answered easily. The perceiving what is intimate depends on several factors.
In general it has to be differentiated in the culture, how a human is perceiving the self and what is shaping the sociocultural live \cite{carrithers1985category}. It is not possible to consider all well-known cultures in this work, therefore the focus is limited on the scrutiny of the western civilization. 

In the western civilization privacy takes up a lot of space. Nevertheless, the state of a person in the society is defining the personal perceiving of privacy and intimate data. And the personal view, as well.
These things can not be defined in a few sentences, the topic is to complex and not measurable. Furthermore, it is subjective. For the individual, the perception of intimate data is different.

Due to this, the definition of what is perceived as intimate for people living in the western civilization, will be shown by the following examples. Several work are focused on intimate date in in different contexts. Although, a clear definition of what data is intimate or is what people perceive as intimate is not found.
In the following some descriptions are summarized to give a rough outline.

The focus in Danaher et al. \cite{doi:10.1080/15265161.2017.1409823} is on intimate interpersonal relationships. In this work there is no clear definition to be found. They think a definition for such term is not needed. However, to describing a romantic relationship the authors in this work are writing the following:

\begin{quote}
	[...] we trust that most readers' intuitive sense of those terms [..] will be adequate for our arguments to make sense". 
	That said, romantic relationship might usefully be thought of as a cluster concept, with paradigmatic examples in the middle, and less paradigmatic examples clustered around it, each one different along various dimensions (e.g., the degree to which sexual interaction is central to the relationship).
\end{quote}

So, if they can define an intimate or romantic relationship about such a way, this concept will also fit for the term intimate. The idea to use an cluster concept can be thought of one step further. The sensitivity or level of intimate data could be arranged in some sort of data hierarchy.