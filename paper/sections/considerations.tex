\section{Consideration of each condition in life course}
\label{sec:consideration_life_course_conditions}

\textit{TODO: For each state, the results from the individual papers are collected.
Further individual distinctions can be derived from this (for instance in B it is is to be distinguished between intimate tracking and intimate gamification).
The types of possible data collection, tracking and also sharing is to be reported for each condition in life course.}

\subsection{Condition A: Dating: Scoping out potential intimates}

At the beginning of a potential relationship we want to know more about the person of our interest. Due to this, we so called collect data about this person. A good way the get relevant information is using a standard social network like Facebook \footnote{\url{www.facebook.com}} or using Google search. Monitoring a person on Facebook is known as Facebook stalking \cite{levy2014intimate}. To stalk another person on Facebook undiscovered, much articles has been written about \cite{sueddeutsche_fb_stalking}. With the Website stalkscan.com \footnote{\url{https://stalkscan.com}} it is possible to get all public entries from a persons Facebook profile site which is public by only one mouse click. Surley, it can only shows what is already set public, still it make it more easily to talk another person very quickly.
Within this website as tool is also avoided to give an involuntary like by clicking through the photographs, for instance.
The Google search mentioned at first is known as \textit{google someone}. With this method is it possible to get information from every source which is findable for the search machine \cite{nolan2005hacking}. Also for this topic there are many article how to \textit{google someone}. For instance, the search on images is of high interest \footnote{\url{https://www.lifewire.com/google-people-search-3482686}}.

The topic in this condition A is not only searching data about someone, but also create such data. Levy \cite{levy2014intimate} mentioned the application Lulu as a tool to create data for use in prospective relationships. The focus of this application is on campus life. The app Lulu gives young women the opportunity to review male students and friends, with which they are connected on Facebook. The review contains information in relation to humor, manners, look and style, sex and kissing. The review giving by the female users is anonymously. In the first version, each male fiend on Facebook could be reviewed in this app. But after concerns related to privacy of reviewed male Facebook users, a review can only be committed for such male user which have explicitly allowed to this.

Furthermore, such services that combine online dating with user's geographical location are well known. Tinder is a widespread location-based dating service. The app shows potential people with different interests (e.g. romantic relationship) near to the user's location or next holiday destination \footnote{\url{https://tinder.com}}. By showing the user several profiles he/she can deside to swip right for a like. If the other person does also a right swip, it is a match. Now the user can exchange messages, for instance to get a date. The princip sound easy, but isn't at all. By using these app, a huge among of intimate data is collected. 
First of all, the tinder app is connected to Facebook and Instagram, a photo-sharing social networking service, owned by Facebook itself. In order to this there is a huge commercial interest to assume. Judith Duportail demanded access to here personal data under European data protection law after four year using the tinder app. The respons was an over 800 site report containing diffrent types of data like Facebook likes, information about education, age-rank of men she was interested in, number of Facebook friends, when and were every online conversation with her matches happened, also interests and jobs, pictures, sexual preferences. The list contains a huge amount of intimate data. In the article Duportail writes, she was amazed by how much information she was voluntarily disclosing. This was also called secondary implicit disclosed information. Firms have an increasing interest in gathering personal data from user's activities \cite{taylor2009privacy}. This results in a trade-off for the user - use the system and accept privacy concerns due to the commercial interest from the provider, or abstain the service.
Nevertheless all concerns, users reveal their data very quickly, as shown in Tait et al. \cite{tait2015hello}. 
Users who tend to gain confidence quickly, therefore, also more quickly reveal more information. In addition, this study showed that higher profile activity increases the amount of information desired. 
That means, users who maintain an active profile and present activity also receive more and higher information from other user's rather than users of profiles that provide barely information. The disclosure of information is determined in part by the personality of the user and the context in general. This affects how users surround their data online and with strangers. They found out that in only 6 - 10 minutes a user can extract the full name and date of birth from a conversation. Within these information it is easy to get further data about the person via Google search and Facebook, for instance.

In Nandwani et al. \cite{10.1007/978-3-319-61542-4_32} it was examined how quickly users reported their data to strangers and, above all, which data. For the study, an automatism was developed to contact 100 Tinder users. The study was a single blind study, so users did not know at the moment that they were writing with a Chat-bot. The evaluation of the data yielded the following results: Most of the published data was personal data, for instance: full name, date of birth, phone numbers, work details, email-addresses, complete address and other data that will not be listed here.

Why are this data disclosed to strangers in online platforms and apps? As mentioned above, the user trusts in the authenticity of the other within an active profile account. Also they do not reflect the inpacts of disclosure there personal and also intimate data. For this purpose, Nandwani et al. \cite{10.1007/978-3-319-61542-4_32} suggest an virtual assistant in such applications like Tinder, which study the relationship between the users by parameters and inform the user which information should be reveal in the conversation.
\begin{description}
	\item[A] \textit{Data collection at the beginning of a relationship, Facebook stalking, potential partner googling, Tinder. In the following: why is this used or why are these data collected, recorded etc. Subsequently, how do people perceive this, influence of data on perception}
	\item[B] \textit{Categorization in intimate tracking and intimate gamification from \acl{QR}: example of these apps and tracking devices. What added value do they have in the relationship? What's in it? How do people perceive that (Quantifying, over-trust in numbers).}
	\item[C] \textit{Drafting the role of women at this stage of a relationship: many apps and devices for tracking women (cycle, fertility, etc.).}
	\item[D] \textit{Category intimate surveillance from \acl{QR}: main emphasis:
	Tracking the partners in a relationship: acceptable or not by mutual agreement? Does that affect the relationship, or the mutual trust? There is no investigation until now (continue at the end (conclusion, further work)).}
\end{description}