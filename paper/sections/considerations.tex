\section{Consideration of each condition in life course of intimate relationships}
\label{sec:consideration_life_course_conditions}
This sections is about how technologies like social networks, search engines, tracking devices and applications for the smart phone may be applied in a relationship.
For that each condition is investigated for the type of data that can be searched, collected, tracked and shared by the user or others.
The conditions are treated in separate paragraphs, which summarize the intimate data that may are collected, tracked and monitored. Also it is about how these data are perceived by the user and how the data affect the users perception.

\subsection{Condition A: Dating - Scoping out potential intimates}
\label{subsec:A}
At the beginning of a potential relationship one want to know more about the other person of one's own interest. Due to this, one collect data about this person. 
\subsubsection{Searching for information}
A way the get relevant information about another person of one is interested in is to use a standard social network like Facebook \footnote{\url{www.facebook.com}} or a search engine like Google Search \footnote{\url{www.google.com}}.
Monitoring a person on Facebook is known as Facebook stalking \cite{levy2014intimate}. To stalk another person on Facebook undiscovered, much articles has been written about \cite{sueddeutsche_fb_stalking}. With the Website stalkscan.com \footnote{\url{https://stalkscan.com}} it is possible to get all public entries from a persons Facebook profile site by only one mouse click. While it can only show what is already public, still it makes it easier to stalk another person quickly.
With this website as a tool to spy on information about another person avoids making an involuntary like when clicking through the photos on Facebook.
With the Google Search one can find information about another person which are available on the web, as mentioned above. This is commonly known as \textit{google someone}. With this method is it possible to get information from every source which is findable for the search engine \cite{nolan2005hacking}. There are many article how to \textit{google someone}. In addition, there are suggestions from Google Search itself to find what to do, as sown in figure \ref{fig:how_to_google_someone}.

\begin{figure}[htb]
  \centering
  \includegraphics[width=\linewidth]{img/how_to_google_someone.jpeg}
  \caption{Google Search suggestions for google someone, [Source: screen shot toke on 07.07.2018 on \url{www.google.com}}.
  \label{fig:how_to_google_someone}
\end{figure}

\subsubsection{Creating and providing information}
The topic in condition A is not only searching for data about another person, but also create such data. Levy \cite{levy2014intimate} wrote about the application Lulu as a tool to create data for use in prospective relationships. The focus of this application is on campus life. Lulu gives young women the opportunity to review male students and friends, with which they are connected on Facebook. The review contains information in relation to humor, manners, look and style, sex and kissing. The review giving by the female users is anonymously. In the first version of Lulu, users could review each male friend which they had on Facebook. But after privacy concerns by the reviewed male Facebook users, a review can only be committed for such male user which have explicitly allowed to this.

Further on, services that combine online dating with user's geographical location are well known. Tinder is a widespread location-based dating service. The smartphone application shows potential "dates" or partners with common interests (e.g. romantic relationship) near to the user's location or next holiday destination \footnote{\url{https://tinder.com}}. By showing the user several profiles he/she can decide to swip right for a like. If the other person does also a right swip, it is a match. Now the user can exchange messages, e.g. to arrange a date. The principle sound easy, but isn't at all. By using the tinder application, a huge amount of intimate data are collected. 
Tinder is connected to Facebook and Instagram, a photo-sharing social networking service, owned by Facebook itself. In order to this there is a huge commercial interest to assume.

Duportail, a French journalist, demanded access to here personal data after four year using the tinder application \cite{taylor2009privacy}. In the \acs{EU}, she has the legal means to request this data, using the European data protection law. The response was an over 800 site report containing different types of data like Facebook likes, information about education, age-rank of men she was interested in, number of Facebook friends, when and were every online conversation with her matches happened, also interests and jobs, pictures, sexual preferences. The list contains a huge amount of intimate data. In her article Duportail wrote that she was surprise by how much information she was voluntarily disclosing.
In \cite{taylor2009privacy} this is called \textit{secondary implicit disclosed information}. Firms have an increasing interest in gathering personal data from user's activities. This results in a trade-off for the user - use the system and accept privacy concerns due to the commercial interest from the provider, or abstain the service.

Nevertheless all concerns, users reveal their data quickly, as shown in Tait et al. \cite{tait2015hello}. 
The study showed that users who tend to gain confidence quickly, therefore, also more quickly reveal more information. In addition, it showed that higher profile activity increases the amount of information desired. 
That means, users who maintain an active profile and present activity also receive more and higher information from other user's rather than users of profiles that provide barely information. The disclosure of information is determined in part by the personality of the user and the context in general. This affects how users surround their data online and with strangers. They found out that in only 6 - 10 minutes a user can extract the full name and date of birth from a conversation. Within these information it is easy to get further data about the person via Google search and Facebook.

In Nandwani et al. \cite{10.1007/978-3-319-61542-4_32} it was examined how quickly users reported their data to strangers and, above all, which data. For the study, an automatism was developed to contact 100 Tinder users. The study was a single blind study, so users did not know at the moment that they were writing with a Chat-bot. The evaluation of the data yielded the following results: Most of the published data was personal data, for instance: full name, date of birth, phone numbers, work details, email-addresses, complete address and other data that will not be listed here.

This data were disclose to strangers in online platforms and applications, due to the fact that the user trusts in the authenticity of the other within an active profile account. Also they do not reflect the impacts of disclosure the personal and also intimate data. For this purpose, Nandwani et al. \cite{10.1007/978-3-319-61542-4_32} suggest an virtual assistant in such applications like Tinder, which analyze the relationship between the users by parameters and inform the user which information should be reveal in the conversation.

\begin{table*}[t]
	%% Table captions on top in journal version
	\caption{Interrelated types of intimate data in \acl{QR}, that can be tracked in a romantic relationships [Source: table content from \cite{doi:10.1080/15265161.2017.1409823}, \cite{levy2014intimate} and \cite{doi:10.1080/13691058.2014.920528}}
	\label{tab:typ_of_QR}
	\scriptsize
	\begin{center}
		\begin{tabular}{|p{4cm}|p{11cm}|p{2cm}|}
			\hline
			Type & Description & Examples \\
			\hline
			\hline
			\textbf{Intimate tracking} &  Collection of all (measurable) data that can arise through intimate behaviors (in a relationship), e.g. number of partners, number of sexual encounters, duration of sexual encounter, or romantic behaviors (gifts, help in the household, attention) & SexTracker \newline SexKeeper \newline Nipple \newline Lovely \newline kGoal \\
			\hline
			\textbf{Intimate gamification} & Use of gamelike incentives to change or improve the behavior in a romantic relationship; Playful learning to lead a successful relationship & Glow Application \\
			\hline
			\textbf{Intimate surveillance} & Use of technologies to monitor intimate partners & Find my Friends \\
			\hline
		\end{tabular}
	\end{center}
\end{table*}

\subsection{Condition B: Tracking Intimate Practices}
\label{subsec:B}
The potential of creating, collecting and tracking intimate data rises if the romantic relationship between two individual deepens. Such a relationship in which intimate data are tracked is named a \ac{QR} by Danaher et al. \cite{doi:10.1080/15265161.2017.1409823}.
The authors described in their work three categories of intimate data which can be tracked in a \acs{QR}. In table \ref{tab:typ_of_QR} the three categories are summarized with a descriptions and examples.

In the following the categories intimate tracking and intimate gamification are considered in more detail.
The third category intimate surveillance will be discussed in section \ref{subsec:D}.

\subsubsection{Intimate Tracking}
For tracking intimate data a variety of applications are available that provide multiple functions. These applications usually track a huge amount of data about sex activities, e.g.the number of partners, the number of "sessions" per partner, the sexual positions used during theses sessions, the number of thrusts per session, duration of these sessions, number of calories burned per session, and so on \cite{doi:10.1080/15265161.2017.1409823}. 
This list only mentions the most common. There are many more variants of intimate data that can be tracked. As Kelly wrote in \cite{kelly2017inevitable}, nearly everything that can be measured is tracked nowadays. Maybe this does not cover the large amount of users, but this possibility is still used.

The data are voluntarily or automatically tracked using such technologies \cite{doi:10.1080/15265161.2017.1409823}. That said, data are either actively provided by users through activating functions like recording sound or automatically recorded, e.g. by running the application in the background of the smartphone. Maybe the user is not aware what is recorded all the time. 

However, it is not only possible to track the data, but also to share it with others to compare or measure with like-minded people.
This is also referred to as \textit{participatory surveillance}. As Lupton \cite{doi:10.1080/13691058.2014.920528} writes, this includes looking at oneself, but for one's own purpose. Self-tracking is often associated with self-reflection, but it has less to do with it \cite{lupton2016quantified}. Rather, it is a visualization and reflection of the collected numbers. But the reflection of the self in this context involves much more than the visualization of the numerical data. It is more of a strict focus on the pure numbers. These numbers are only objectively perceived, and no longer associated with the subjective activity or context to which they once belonged.
Often, these applications also contain elements for the gamification of the mission or goals.

\subsubsection{Intimate Gamification}
Another observation is the gamification in this area of tracking. Users are encouraged to quantify their sex life in order to measure their performance and compare themselves with other users \cite{doi:10.1080/13691058.2014.920528}. This type of quantification mainly focuses on the male user.

One consequence of using such technologies may be the reinforcement of gender stereotypes, as Lupton wrote in \cite{doi:10.1080/13691058.2014.920528}. The algorithm defines the goals by which users orient and measure themselves. The individuality may be lost with it.

In addition, this kind of feedback does not necessarily have to be of good quality and a lasting effect on interpersonal relationships. \cite{doi:10.1080/15265161.2017.1409823}. A good relationship is not measured by how much sex one have or how long it lasts. As explained in the section \ref{sec:terms_of_definition} above, each relationship is individual and to complex for only rating by numbers. There are other intrinsic values that make a good relationship.

\subsubsection{Intimate Surveillance}
As mentioned in the beginning of these section, surveillance in the life course of a relationship is considered in a much detailed way in section \ref{subsec:D}.

\subsubsection{Objections}
The automatic recording of such data in an application can be very questionable, because the danger is great that the user is not aware of it. Most users do not read the the fine print of the terms and conditions of these services before using them \cite{anaya2018ethical}.

Also, the sole quantification of a relationship does not necessarily lead to an improvement of the relationship skills. Rather, these types of behavioral change supports gender stereotypical reinforcement. That would be a very retrograde development compared to the current perception of our conception of love and sexuality. %TODO: Quelle für diese Behauptung suchen
In addition, as already mentioned above, the users can perceive the data objectively only by quantifying the activities, similar to a sport activity like running. The reflection of the real activity may be lost.

Users share this data with like-minded users, or keep it for themselves and do not share it, or share it with their intimate partner.
Being able to share this data with other users brings a larger audience as before \cite{doi:10.1080/13691058.2014.920528}. This fact also influences the willingness to disclose intimate data to others, mostly strangers.
Users also share the data for the purpose of comparison with other users. In addition, gamification of intimate data is often used in such applications and thus supports the willingness to disclose.

\subsection{Condition C: Monitoring Fertility}
\label{subsec:c}
This section is focused on tracking the menstrual cycle and fertility of female users. These types of data are highly intimate. So far, they have been collected in conjunction with a medical treatment only and evaluated with the 
gynecologist. Nowadays, it is possible to track these data with digital devices and application in several ways.
\subsubsection{Overview of technologies for monitoring fertility}
The menstrual cycle and thus the fertility of the woman has been "monitored" for a long time. The exact beginning is unknown, it has been writing about it since the 1920s in scientific medical \cite{rotzer1988geschichte}. %TODO: Prüfen
With Josef Roetzer the \ac{NCR} and thus the "sympto-thermal method" became well known \cite{roetzer1968erweiterte}. With this method the menstrual cycle could be monitored and thus the fertile days could be determined exactly with a few differences.

In the age of digitalization, there are of course digital technologies that support the female user to monitor ans analyze these data.

The following table \ref{tab:typ_of_app_for_tracking_cycle} lists some (known) applications, which can be found in \textit{Google App Store} or \textit{Apple App Store}.

\begin{table*}[t]
  %% Table captions on top in journal version
  \caption{Examples of applications for tracking the menstrual cycle and determining fertile days, available in summer 2018 [Source: ]}
  \label{tab:typ_of_app_for_tracking_cycle}
  \scriptsize
  \begin{center}
    \begin{tabular}{|p{2cm}|p{3cm}|p{12cm}|}
		\hline
		Term & Operation System  & Description \\
		\hline
		\hline
		
		\textbf{myNFP} &  iOS \newline Android &  Analyze the menstrual cycle according to the sympto-thermal method. All important parameters for evaluation are entered by the user herself \cite{myNFP}
		\\
		\hline
		\textbf{Kindara} Fertility \& Ovulation  & iOS \newline Android & Supports the sympto-thermal method and can be used for \acs{NCR}; Supports also information according to "ovulation pain, sore breasts, acne breakouts"; offers a "community to connect with other users and experts to support" \cite{kindara}\\
		\hline
		\textbf{Lily} & iOS & Evaluated according to the sympto-thermal method or based on average values of other users \cite{lily};
		 \\
		\hline
		\textbf{Glow} & iOS \newline Android & period and ovulation tracker; conceive better understand fertility awareness; "ovulation calculator prediction gets more accurate as you enter more data" \cite{glow}\\
		\hline
	\end{tabular}
  \end{center}
\end{table*}
% month your body tells you a story. Want to hear it?
According to the manufacturer of the \textbf{myNFP} application, the sensitive data is not processed by third parties. Furthermore, as few as possible data is recorded. The data are anonymous and does not indicate the person. The manufacturer justifies this with the argument that the application charges a monthly fee of 2.50 \euro{} \cite{myNFP}.

For the application \textbf{Kindara} can also be found information on privacy, but these look different than in the previous one. An excerpt from the privacy policy provides more information \cite{kindara}:
\begin{quote}
	Kindara collects and uses the information you provide to us when you use the Kindara Service. Information that Kindara may collect includes: name, date of birth, e-mail address, fertility-related data and other family planning and health-related information you provide. You may consider some of this information to be sensitive so you should choose carefully regarding whether and if you will use the Service.
\end{quote}
Since the application is offered for free, it is reasonable to assume that the data will be processed further. The commercial interest of the manufacturer should be noted and analyzed in more detail before using this application.
An additional device is offered for the application, which can be used to measure the wake-up temperature. The device will automatically connect with the application when the temperature is being measured. The data will be sent via bluethoot \footnote{\url{https://www.kindara.com/wink}}.

The \textbf{Lily} application offers its users to whether they use it in full functionality, a contribution will be charged, but therefore, the manufacturer guarantees that the data will not be evaluated by third parties, the personal information will not be stored, and no backup of personal data will be stored on any server \cite{lily}.

\subsubsection{The Glow Application}
\label{subsubsec:glow_application}
The \textbf{Glow} application is described separately here, because other works like  \cite{doi:10.1080/15265161.2017.1409823}, \cite{levy2014intimate} and \cite{doi:10.1080/13691058.2014.920528} also wrote about it.
\begin{figure}[htb]
	\centering
	\includegraphics[width=\linewidth]{img/Glow-App-review-screenshot-1.jpg}
	\caption{The Glow app collects a variety of intimate data (Potograph source: \cite{glowApp})}
	\label{fig:glow_app}
\end{figure}
Launched by PayPal founder Max Levchin in 2013, the application offers great concurrence with many other fertility and natural prevention apps. Glow can track a huge amount of intimate data, e.g. the menstruation, position and firmness of a woman's cervix, sexual intercourse with the women's position during ejaculation, "whether or not they had an orgasm and whether they experienced emotional or physical discomfort during sex" \cite{doi:10.1080/13691058.2014.920528}. In addition, the mood of the user can be tracked.
The difference to other applications is that Glow makes the collection of intimate data a family affair. The users' partners are invited to download a mirror application and provide additional data \cite{levy2014intimate} The application also sends messages to the partner about the current status of the partner's period, reminding of attentions such as flowers or a nice message.
The data of the female users are evaluated collectively in order to be able to specify better forecasts for the individual user from the large collection.

Danaher et al. \cite{doi:10.1080/15265161.2017.1409823} argue under the point \textit{Gender Relationship Objection}, that these types of technologies are making women an object of surveillance and quantification.
They give the impression that the cycle of a woman is unsupervised chaotic and can only be "rebuilt" with strict control.
In addition, the Glow application would promote the development and enhancement of gender stereotypes, as also augmented in \cite{doi:10.1080/13691058.2014.920528}.
 
The disclosure of such intimate data is questionable if the user disregards how the data is further evaluated. These technologies can be helpful in the evaluation of the collected data, and remind of the daily measurement. Unfortunately, these very sensitive data are also used for commercial purposes.

\subsection{Condition D: Surveillance, Abuse and Revenge}
\label{subsec:D}
The condition D is about surveillance in relationships. The other three conditions are also about surveillance, but in a different way. The differences are briefly described and illustrate in the following.
 
The conditions described above deal with the different situations in which intimate data can be created and used, e.g. for surveillance purposes.
The section \ref{subsec:A} covered the collecting of data via social networks and online dating services. In section \ref{subsec:B} the generation and collection of intimate data in a relationship was described. In section \ref{subsec:c} it was discussed about the monitoring of woman or rather their menstrual cycle and fertility.
A summary of the previous conditions can be seen in figure \ref{fig:intimate_surveillance}.
As described above, users voluntarily or unconsciously disclose this data to benefit from data science (see Glow application, which calculates the course of the menstrual cycle among other analytics from the data set of other users, making a relatively reliable prediction of ovulation possible without the user providing daily tracked information).
In all these states one can speak of an \textit{voluntarily participatory surveillance}.
The supervisor is usually the provider of the smartphone application or the wearable devices, which is commercial interested in the data. This possible form of surveillance is discussed critical in more detail in section \ref{sec:risks}.

In this section the mutual surveillance of the partners in existing or also terminated relationships is considered in more detail. 
The type of surveillance in a relationship can be voluntary or involuntary. The threat of providing such kind of intimate data in the context of an intimate relationship should not be disregarded.
\begin{figure}[htb]
	\centering
	\includegraphics[width=\linewidth]{img/differences_surveillance.png}
	\caption{Summarizing of conditions A, B and C, based on the visualization of \cite{ethicsOfSurveillance}}
	\label{fig:intimate_surveillance}
\end{figure}

\subsubsection{Mutual voluntary surveillance provides mutual trust?}
In section \ref{subsec:B} and table \ref{tab:typ_of_QR} the term \textit{intimate surveillance} was already mentioned.
For mutual voluntary surveillance in an existing relationship, Danaher et al. \cite{doi:10.1080/15265161.2017.1409823} have given an interesting but also something questionable approach.

The authors consider if mutual voluntary surveillance in an existing intimate relationship could be useful to provide mutual trust.
First they define the concerns related to the use of such \acs{QR}-technology for supporting partner's mutual trust in a well-functioning relationship. They found, that such a tracking technology could corrode the mutual trust, which a romantic relationship is usually based on.

Levy argue in \cite{levy2014intimate}, that mutual trust in a relationship has played a fundamental role so far and promotes pro-social behavior in the relationship. If digital technologies take on this role now, by tracking the partners in the relationship, and if the partner does control themselves and build their trust on it, it does not rely on loyalty to the partner anymore, but only to the tracking software.

Due to this fact, it is questionable, where the using of such tracking technologies in relationships leads.
Danaher et al. argued that "even if mutual trust is an ideal, it is an ideal that many fall short of in reality." \cite{doi:10.1080/15265161.2017.1409823}.
The authors suggested that partners, to some part in the relationship, voluntarily observe themselves to appease the other's doubts. But they also added some considerations to privacy and security risks. Such use of tracking technology requires extreme caution and respect. It requires the explicitly agreement of the partner. Furthermore, the technology itself should involve a hard-but-reversible lock-in, to bring the surveillance under control and interrupt this if any of the partner would not tracked further from the other.
They suggested that an example could be an smartphone application that allow mutual surveillance, e.g. for a period of time.
It would be interesting to survey if partners would use such an tracking technology, in which circumstances and conditions and, especially, if they would find such an approach desirable and helpful in a relationship. 

Some couples are using tracking technology already in there relationship. %TODO: Quelle
There are a few possibilities to do this via the smart phone. The Google Play Store and also the App Store by Apple offers some applications to locate a person, e.g. friends, a family member or the own children. To give an example, with the application called \textit{Find my Friend} it is possible to locate another person which is also using the same application or (if the other person doesn't use a smartphone) with the agreement via a simple text message \footnote{Find my Friends application: \url{https://play.google.com/store/apps/details?id=com.fsp.android.friendlocator}}. After agreement, the users can communicate and locate each other.
One further option to track people who matter most is given via the operating system of the smartphone itself. Apple offers the service called Family Sharing \footnote{Apples Family Sharing: \url{https://support.apple.com/en-us/HT201087}}. With this service the user can share the actual location with members in the family group after the function called \textit{location sharing} is turned on. With the \textit{Find my Friend} application the user can see the location of each members in the family group, if they share there location too.

Another possibility to share a user's location with friends is the \textit{Live Location feature} of WhatsApp\footnote{Live Location feature of WhatsApp: \url{https://faq.whatsapp.com/de/android/26000049/?lang=en}}. With this function a user can share the location with friends for a period of time.

These three examples should only give an overview what is in use today. It is on no case completely. It only shows that technologies for location tracking are already available and in use.

\subsubsection{With whom do people track their location?}
In \cite{Consolvo:2005:LDS:1054972.1054985} Consolvo et al. investigated the willing disclosure of information from location-enhanced technology users to specific other people. In this study, 16 participants had given a social network consisting of people from their social networks. This network also includes the participants partner, in that work called \textit{spouse/significant other}. They found that for the willing disclosure for the user it is most important "[..] who was requesting, why the requester wanted the participants location, and what level of detail would be most useful to the requester.". The results also shows that "[...] who the requester was had the strongest influence on participants willings to disclosure.". If the partner, called \textit{spouses/significant others} was requesting, the participants "[...] were willing to disclosure something for 93\% of the 670 requests.".
It can be concluded, that people use this technologies and are willing to disclosure informations about theirs location, activities and accompaniment. Further they give more details of information if the have an special relation to the requester, such as a intimate one.

\subsubsection{Why do people track their location with others?}
%TODO. Win more trust?

%TODO: Transparency in relationship?
In the opinion by Ikrath \cite{Ikrath2018} we are living in a change of values. The present generation is non-solidarity and self-centered. Individual values are preferred over community values.  
As a result it could be possibly difficult to have a relationship based on trust, how Danaher et al. already argue in \cite{doi:10.1080/15265161.2017.1409823}.

Also the urge for control could be a possible reason why people are tracking each other.

\subsubsection{Does location tracking corrode the love?}
It is questionable, if location tracking is helpful in a relationship.
Engl et al. \cite{engl2016} actually working on an application with an concomitantly website for parters.
The application should encourage the user to regularly invest time in successful discussions and to improve the communication in the relationship.
In addition, the application gives specified tasks and configurable exercises for reflection and interaction, as well as for assessing the quality of the relationship.
%TODO: Nachfragen, warum?
But the application will be implemented without an location tracking function.

%TODO: Kommunikation

Location traction in a relationship with digital techniques provides some risks regarding to privacy. In the following, the possible dangers are considered in more detail.

\subsubsection{Abuse and revenge}
Unfortunately, such technologies as discussed above can also be misused for other purposes. Freed et al. \cite{freed2018stalker} conducted a study with 89 participants to show how abusers in intimate partner violence context exploit technologies to intimidate, threaten, monitor, impersonate, or harass their victims.
They grouped the different types of attacks by abusers in four categories. A summary of these categories with examples is shown in figure \ref{fig:abusing_categories}.

\begin{figure}[htb]
	\centering
	\includegraphics[width=\linewidth]{img/abusing_categories.png}
	\caption{Summary of different types of attacks by abusers in intimate partner violence context [Visualization based on content from \cite{freed2018stalker}].}
	\label{fig:abusing_categories}
\end{figure}
In the following these four categories grouped by Freed et al. \cite{freed2018stalker} are explained in more detail, focusing on intimate data, which the abuser accessed. Freed et al. found also other attacks (e.g. messages, posts in social networks and phone calls) to harm the victim, but that goes beyond the focus in this work.

\textit{Ownership of Device and Account}
Abuser and victim have or had an intimate relationship. This often includes a cohabitation or  marital togetherness. In such a relationship with shared possession one of the partners commonly is taking responsibility for the couple finances. This fact leads to devices and accounts belonging to the abuser.
The study showed that many participants (n=20) stated that their device (e.g. smart phone) was " [...] bought and paid [...]" by the abuser. 
With the ownership the abuser gained control about the different functions a provider offers to its customers. The abuser was able to control the victims digital accounts, e.g. received the phone bills and therefore knew detail facts about the usage behavior by the victim (including call history, text messages and voice mails), or "[...] exploit the data back-up services [...]" and get information or data, e.g. pictures saved on the victims smartphone. The abuser was also able to use the "[...] location-based services to track victims devices, including anti-theft services (e.g., 'Find My Phone'), parental tracking, and other safety-based services (e.g., Find My Friends') [...]" \cite{freed2018stalker}.

\textit{Account or Device Compromise}
If the abuser was not able to get access through the ownership of devices and accounts, there were also other ways to gain access. The study sowed that "[...] abusers are able to compromise victims' devices or accounts against their will and/or without their knowledge. Such compromises predominantly occurred via two routes: compelled password disclosure and remote compromise of accounts by guessing of victim passwords or the answers to password reset security questions.". If the abuser had access to the device and/or the account, it was no longer difficult to install software for spying the victim.
The study also showed that the "[...] abuser were able to "hack" into victim accounts.". With the access gained by guess the password or compel password disclosure the "[...] abusers used their access to monitor, control, impersonate, or otherwise hurt their victims." \cite{freed2018stalker}.

\textit{Harassing and Threatening Messages and Posts}
Abuser used social networks to harm their victims. The victims were harass with messages or calls on the smartphone device. In addition, the networks were used to damage the victims' reputation.
Therefore abuser contacted friends and family of the victims in order to negatively influence the friendship and/or to pursue their jealousy.

\textit{Exposure of Private Information}
Digital technologies offer abusers a way to harm their victims by disclosure private information to third parties or friends and social contacts. Freed et al. found that "[t]he most common exposure-based threat [...] was exposure of intimate images (photos or videos) of victims, commonly known as non-consensual pornography or \textit{"revenge porn"}[...].

Levy also wrote about the \textit{revenge porn} as an possible risk in condition D in the life course of intimate relationships.

Tong surveyed in \cite{Tong2013Facebook} "[...] how [...] individuals use Facebook as a from of surveillance." The survey showed that there were three dimensions for general social activity monitoring. The first was "[...] referred to looking at the ex's profile, photos, and status updates to see what the ex is doing." Second it was important to "[...] detecting an ex-partner's new romantic interests [...]", e.g. by checking the realtionship status of the ex-partner. The third dimension included "[...] direct statements made to, or by the ex-partner [...].
Compared to the results in the study from Freed et al. \cite{freed2018stalker} mentioned above these three dimensions seems harmless and quite safely. However, these activities are also kinds of surveillance that the user, which is monitored, can not controlled. It is similarly to the research results in condition A in \ref*{subsec:A}.

In summary, partners in intimate relationships are increasingly discovering digital technology to harm their partner. Freed et al. \cite{freed2018stalker} found that the attacks were technologically unsophisticated and often carried out by a \textit{UI-bound adversary}.









