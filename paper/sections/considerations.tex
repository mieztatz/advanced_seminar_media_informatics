\section{Consideration of each condition in life course}
\label{sec:consideration_life_course_conditions}

\textit{TODO: For each state, the results from the individual papers are collected.
Further individual distinctions can be derived from this (for instance in B it is is to be distinguished between intimate tracking and intimate gamification).
The types of possible data collection, tracking and also sharing is to be reported for each condition in life course.}

\subsection{Condition A: Dating: Scoping out potential intimates}

At the beginning of a potential relationship we want to know more about the person of our interest. Due to this, we so called collect data about this person. A good way the get relevant information is using a standard social network like Facebook \footnote{\url{www.facebook.com}} or using Google search. Monitoring a person on Facebook is known as Facebook stalking \cite{levy2014intimate}. To stalk another person on Facebook undiscovered, much articles has been written about \cite{sueddeutsche_fb_stalking}. With the Website stalkscan.com \footnote{\url{https://stalkscan.com}} it is possible to get all public entries from a persons Facebook profile site which is public by only one mouse click. Surley, it can only shows what is already set public, still it make it more easily to talk another person very quickly.
Within this website as tool is also avoided to give an involuntary like by clicking through the photographs, for instance.
The Google search mentioned at first is known as \textit{google someone}. With this method is it possible to get information from every source which is findable for the search machine \cite{nolan2005hacking}. Also for this topic there are many article how to google someone. For instance, the search on images is of high interest \footnote{\url{https://www.lifewire.com/google-people-search-3482686}}.

\begin{description}
	\item[A] \textit{Data collection at the beginning of a relationship, Facebook stalking, potential partner googling, Tinder. In the following: why is this used or why are these data collected, recorded etc. Subsequently, how do people perceive this, influence of data on perception}
	\item[B] \textit{Categorization in intimate tracking and intimate gamification from \acl{QR}: example of these apps and tracking devices. What added value do they have in the relationship? What's in it? How do people perceive that (Quantifying, over-trust in numbers).}
	\item[C] \textit{Drafting the role of women at this stage of a relationship: many apps and devices for tracking women (cycle, fertility, etc.).}
	\item[D] \textit{Category intimate surveillance from \acl{QR}: main emphasis:
	Tracking the partners in a relationship: acceptable or not by mutual agreement? Does that affect the relationship, or the mutual trust? There is no investigation until now (continue at the end (conclusion, further work)).}
\end{description}