% Move 1: Establishing a Territory

% Step 1: Claiming importance and/or
% Step 2: Making topic generalization(s) and/or
% Step 3: Reviewing items of previous research
In the century of digitalization there are many opportunities offered to perceive the self in everyday life in a different way as before. Tracking and quantifying the self, the body an also other aspects in life is commonly used as showed in \cite{kelly2017inevitable}. Nowadays many people are engaged in tracking such data like heartbeat, sleeping pattern and other quantifiable data. They are tracking and also sharing this information with others, like friends or like-minded people. 

% Move 2: Establishing a Niche

% Step 1A: Counter-claiming or
% Step 1B: Indicating a gap or
% Step 1C: Question-raising or
% Step 1D: Continuing a tradition

But there are many different types of data which can be tracked. Superficially, such data like heartbeat or sleeping pattern does not seem apprehensive when tracking and sharing these with others, but how about data in intimate relationships and sexual behaviors?

% Move 3: Occupying the Niche

% Step 1A: Outlining purposes or
% Step 1B: Announcing present research
In this work it is investigated how people use techniques for collecting, tracking, storing and sharing of intimate data in romantic relationships. Techniques in this field also called \acl{QR} techniques. In addition to tracking and sharing, surveillance also plays a key role, which is considered.
Therefore, the following questions will be answered by reviewing literature and studies in this scientific field:
%TODO: Answering these questions in conclusion
 \begin{description}
 	\item[RQ1:] What data is perceived as intimate? In what circumstances?
 	\item[RQ2:] Why do people track intimate data in relationships?
 	\item[RQ3:] What do they do with, e.g. tracking, storing, sharing and discussing and with whom?
 	\begin{enumerate}
 		\item Do they over-trust the tracked data?
 		\item How do they perceive their tracked data?
 	\end{enumerate}
 \end{description}
For answering the questions mentioned above a research of literature and studies on collecting, tracking and sharing intimate data in romantic relationships is carried out. In Addition, some articles and reports by users of tracking and other technologies are also used for finding answers, e.g. the report by Duportail \cite{Duportail2017} about the data which tinder collect over a period of time.

% Step 2: Announcing principal findings
Intimate data are searched, collected, tracked, stored and shared in every part of relationship, from the beginning until to termination.
The different types of intimate data were collected from \textit{Facebook} and \textit{Tinder} and tracked by \acl{QR} technologies, for example to quantify sexual activities or measuring menstrual cycle.
The several types of intimate data are obtained from different sources. At the beginning of a relationship, data are gathered from Facebook, Tinder and other soical networks.
At the point the relationship is "established", other technologies become more relevant, e.g. \acl{QR} technologies, to quantify sexual activities or measuring menstrual cycle of the partner, to gain more details about the mood.
If a romantic relationship is broken off, social networks and \acl{QR} technologies can be abused, e.g. to stalking the ex-partner, or even harm. 
The usage of all these digital helpers in intimate relationships is always associated with a certain risk.
Often, default security precautions are insufficient for specific situations, e.g. as in the dissolution of a relationship. 
Frequently users are not aware about the type and scope of intimate data which collected and tracked by their own devices (e.g. the smart phone), and which my be accessible by others like the ex-partner or third parties.

% Step 3: Indicating research article structure
In section \ref{sec:terms_of_definition} the term \textit{intimate} is defined by gathering different definitions related \acl{QR}-technologies and intimate surveillance.
In section \ref{sec:life_course} the so called life course of intimate data is defined including four conditions in which an intimate relationship could be.
The following section \ref{sec:consideration_life_course_conditions} describes the conditions with regarding to the intimate data that are searched for, tracked, stored and shared in these circumstances.
In section \ref{sec:risks} the risks related to the use of such technologies in a romantic relationship is investigated.
Section \ref{sec:conculsion} includes a summarization of the work with a short view for future steps.