% Move 1: Establishing a Territory

% Step 1: Claiming importance and/or
% Step 2: Making topic generalization(s) and/or
% Step 3: Reviewing items of previous research
In the century of digitalization there are many opportunities offered to perceive the self and own life in a different way as before. Tracking and quantifying is commonly used. Nowadays many people are engaged in tracking their data. They are tracking and also sharing this information with other people, like friends or like-minded people. 

% Move 2: Establishing a Niche

% Step 1A: Counter-claiming or
% Step 1B: Indicating a gap or
% Step 1C: Question-raising or
% Step 1D: Continuing a tradition

But there are many different types of data, which can be tracked. Such data like heartbeat or sleeping pattern does not seem to be too intimate when tracking and sharing, but how about data in intimate relationships and sexual behaviors? 

% Move 3: Occupying the Niche

% Step 1A: Outlining purposes or
% Step 1B: Announcing present research
In this work, the collecting, tracking, storing and sharing of intimate data in romantic relationships is investigated.
Therefore, the following questions will be answered by searching for  literature and studies in this scientific field:
 \begin{enumerate}
 	\item What data is perceived as intimate? In what circumstances?
 	\item Why do people track intimate data in relationships?
 	\item What do they do with, e.g. tracking, storing, sharing and discussing and with whom?
 	\begin{enumerate}
 		\item Do they over-trust the tracked data?
 		\item How do they perceive their tracked data?
 	\end{enumerate}
 \end{enumerate}
For answering of the mentioned above questions a research of literature and studies on collecting and tracking intimate data in romantic relationships is carried out. The answering of the first question is not as easy as it seems. Therefore, several definitions from different source are collected.

% Step 2: Announcing principal findings
Intimate data are searched, tracked, stored and shared in every kind of relationship, from the beginning until to termination.
The most different types of intimate data are collected from Facebook and Tinder and tracked by technologies for sexual activities and measuring a woman's cycle.
% Step 3: Indicating research article structure

In section \ref{sec:terms_of_definition} the term \textit{intimate} is defined by gathering different definitions related \acl{QR}-technologies and intimate surveillance.
In section \ref{fig:live_course} the so called life course of intimate data is defined including four conditions in which an intimate relationship could be.
The following section \ref{sec:consideration_life_course_conditions} describes the conditions with regarding to the intimate data that are searched for, tracked, stored and shared in these circumstances.
In section \ref{sec:risks} the risks related to the use of such technologies in a romantic relationship is investigated.
Section \ref{sec:conculsion} includes a summarization of the work with a short view for future steps.

%\section{Related Work}
%\label{sec:relatedWork}
%\cite{doi:10.1080/13691058.2014.920528} shows an critical analysis of a special type of digital health care devices, such as technologies for tracking and sharing users sexual and reproductive activities and functions. Some of these smart-phone apps available in the Apple App Store and also the Google Play Store are investigated of their sociocultural, ethical and political implications. It is shown that such applications emerging ethical and privacy implications, e.g. perpetuate nominative stereotypes and assumptions. It suggests a queering of such technologies and their use.
%
%In \cite{doi:10.1080/15265161.2017.1409823} the research is focused on \ac{QR}. In this work a detailed ethical analysis is provided. The authors found eight core objections to \acs{QR} and investigated these critical. They found out that despite criticism the \acs{QR} tracking technologies can be rated as helpful to support intimate relationships.
%
%In \cite{sjoklint2015complexities} the interplay of technology, data and the self-experience while using self-tracking technology is been investigated. In this work, technology for tracking movement and sleeping activities was used. The investigation shows, that using self tracking devices are not necessary to change behaviors. It is more useful as a re-focusing device. It also shows that the user experiences tends between rational and emotional behaviors when reflecting the tracked data.
%
%In \cite{choe2014understanding} 52 video interviews were taken to understand users, which tend to use self-tracking technologies more than other people do, like share best practices and mistakes through talks, blogging and conferences. The topic of the interview includes, how the users did track themselves and what did they learn. Furthermore, in this work several common pitfalls to self-tracking were found. At the end it is suggested for future research on these pitfalls.