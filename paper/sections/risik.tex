\section{Risks}
Lupton writes in \cite{doi:10.1080/13691058.2014.920528}:
\begin{quote}
	"Now that mobile digital technologies that can be used for surveillance are part of everyday social life.
\end{quote}
Since the technologies discussed above are in daily use, they pose some risks to the users privacy, the perception of themselves and also of the relationship which they lead.
In this section these risks are summarized to give an overview.
The overview is divided into the three categories quantification, trust in a relationship and user privacy.

\subsubsection{Quantification: Perception and rating of the self and the relationship}
Due to the various ways in which intimate data can be tracked, there is a risk of losing the actual reference to the data \cite{doi:10.1080/13691058.2014.920528} and \cite{lupton2016quantified}. In condition \ref{subsec:B} it was mentioned that by tracking of sexual activities the actual act later is only perceived by numbers, thus the act is quantified. The quality or actual perception by the user can be lost. Or put another way, the user is lost in a jumble of numbers \cite{kelly2017inevitable}.
When using these technologies, the user should be aware of why he or she is using them and what these data are actually collected for \cite{doi:10.1080/15265161.2017.1409823}. 
Often it is the case that many users are interested in tracking at the beginning, but after a while they give up using the tracking device and are no long interested in \cite{sjoklint2015complexities}.

\subsubsection{Trust: unknowingly and knowing tracking by intimate partner; over-trust in data only}

\subsubsection{Users privacy risks}
%\begin{enumerate}
%	\item \textit{Quantification (perception and rating of the self and the relationship)}
%	\item \textit{Trust (unknowingly tracking by intimate partner, over-trust in data only)}
%	\item \textit{Privacy (risks, current news, data gaps, etc.)}
%\end{enumerate}