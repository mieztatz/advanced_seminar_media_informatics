\section{Risks}
Lupton writes in \cite{doi:10.1080/13691058.2014.920528}:
\begin{quote}
	"Now that mobile digital technologies that can be used for surveillance are part of everyday social life.
\end{quote}
Since the technologies discussed above are in daily use, they pose some risks to the users privacy, the perception of themselves and also of the relationship which they lead.
In this section these risks are summarized to give an overview.
The overview is divided into the three categories quantification, trust in a relationship and user privacy.

\subsubsection{Quantification: Perception and rating of the self and the relationship}
Due to the various ways in which intimate data can be tracked, there is a risk of losing the actual reference to the data \cite{doi:10.1080/13691058.2014.920528} and \cite{lupton2016quantified}. In condition \ref{subsec:B} it was mentioned that by tracking of sexual activities the actual act later is only perceived by numbers, thus the act is quantified. The quality or actual perception by the user can be lost. Or put another way, the user is lost in a jumble of numbers \cite{kelly2017inevitable}.
When using these technologies, the user should be aware of why he or she is using them and what these data are actually collected for \cite{doi:10.1080/15265161.2017.1409823}. 
Often it is the case that many users are interested in tracking at the beginning, but after a while they give up using the tracking device and are no long interested in \cite{sjoklint2015complexities}.

\subsubsection{Trust: unknowingly and knowingly tracking by intimate partner; over-trust in data only}
The surveillance of the partner without his consent is on the topic of \acs{QR}-Technologien out of the discussion, as Danaher et al. in \cite{doi:10.1080/15265161.2017.1409823} argue. This is clearly the abuse of the data. This includes also the use of such apps as Flexispy \footnote{\url{https://www.flexispy.com/en/}}.

However, the approach that partners voluntarily monitor each other as described in D could also create problems related to the use of such an app or tracking device. This includes for instance the abuse by a dominant partner that might force the use of such software in the relationship. That would not be mutual agreement.

\subsubsection{Users privacy risks}
In Danaher et al. \cite{doi:10.1080/15265161.2017.1422294} the risks associated with the use of \acs{QR}-technologies are summarized.
The authors argumented that the concerns " [...] of the privacy-invading elephant lurking in the room [...]" are not alone a problem of a single person, but also involves one or more persons. However, this is exclusively private and an interpersonal matter, for instance if ever and with what device \acs{QR} technologies are used.
As further concerns, they stated that users use apps on devices that are also used for other purposes, such as smart phones, and that these devices are connected to the Internet. 
They conclude their argument that it is not a single process of tracking the data, which leads to problems. Rather, the problem lies in the fact that third parties collect the data on the devices that track the data, and so they get the data within existing network connection.
Remedy would create devices that only track without transmitting the data.
It used to be tracked without digital helpers, see  \ref{sec:c} the symptothermal method model from Roetzer.
%\begin{enumerate}
%	\item \textit{Quantification (perception and rating of the self and the relationship)}
%	\item \textit{Trust (unknowingly tracking by intimate partner, over-trust in data only)}
%	\item \textit{Privacy (risks, current news, data gaps, etc.)}
%\end{enumerate}