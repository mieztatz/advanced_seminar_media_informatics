\section{Risks}
\label{sec:risks}
Lupton wrote in \cite{doi:10.1080/13691058.2014.920528} "[...] that mobile digital technologies that can be used for surveillance are part of everyday social life."
Since the technologies discussed in section \ref{sec:consideration_life_course_conditions} are in daily use, they pose some risks to the users privacy, the perception of themselves and also of their relationships.
In this section some of these risks are summarized to give an overview.
The overview is divided into three categories which are described in the following.

\subsection{Quantification: Perception and rating of the self and the relationship}
Due to the various ways in which intimate data can be tracked, there is a risk of losing the actual reference to the data as Lupton wrote in  \cite{doi:10.1080/13691058.2014.920528} and \cite{lupton2016quantified}.

In Condition B in \ref{subsec:B} it was mentioned that by tracking of sexual activities the act itself is only perceived by numbers at a later time, thus the act is quantified. The quality and perception of scenes felt by the user can be lost. Or in other words, the user can be lost in a jumble of numbers \cite{kelly2017inevitable}.
When using these technologies, the user should be aware of why he or she is using them and what these data are actually collected for \cite{doi:10.1080/15265161.2017.1409823}. 
Often it is the case that many users are interested in tracking at the beginning, but after a while they give up using the tracking device and are no long interested \cite{sjoklint2015complexities}.

In Condition C in \ref{subsec:c} the tracking of the cycle and fertility of female users is described. 
Especially for the sympto-thermal method by Roetzer a digital device to support the measurement and evaluation of the measured values could be helpful. However, it is claimed that the 
analog measurement leads to better results \cite{roetzer1968erweiterte}.
Regardless of this, there is also the possibility of losing in the tracked data and not paying attention to one's own body feeling.

This also applies to applications in which advice and tips for the relationship are given in the form of notifications, e.g. by the Glow application mentioned in \ref{subsubsec:glow_application}. It is questionable whether this type of support for the relationship is sustainable or whether it is influencing the self-questioning of the actual relationship status.

\subsection{Trust: Unknowingly and Knowingly Tracking by Intimate Partner}
The risk of being lost in data also applies to data obtained through mutual (location) tracking in relationships described in Condition D in \ref{subsec:D}. This kind of tracking also includes the risk that the focus is solely on the data and the trust on which relationships normally build up is lost. The fact that one knows the exactly location of the partner at every time can lead to wanting constant knowledge about the partners location. What is if the location data is not available once a time? This should be examined in future work.

%The surveillance of the partner without his consent is on the topic of \acs{QR}-Technologien out of the discussion, as Danaher et al. in \cite{doi:10.1080/15265161.2017.1409823} argue. This is clearly the abuse of the data. This includes also the use of such apps as Flexispy \footnote{\url{https://www.flexispy.com/en/}}.

The approach that partners voluntarily monitor each other as described in D could also create problems related to the use of such an application or tracking device.
This includes for instance the abuse by a dominant partner that might force the use of such software in the relationship.
\begin{quote}
Many survivors were unaware that their location could be tracked using these services and asked us to teach them how to turn off location services on their phone. Professionals also described how survivors' lack of awareness regarding location tracking may result in potentially dangerous physical stalking [...]. \cite{freed2018stalker}.
\end{quote}


\subsection{Privacy: Risks related to \acs{QR} technologies}
In Danaher et al. \cite{doi:10.1080/15265161.2017.1422294} the risks associated with the use of \acs{QR}-technologies are summarized.
The authors argued that the concerns " [...] of the privacy-invading elephant lurking in the room [...]" are not alone a problem of a single person, but also involves one or more persons, e.g. a person with which one is sharing knowledge about intimate facts like such in a relationship. In this case it is a exclusively private and an interpersonal matter. The decision whether and with what device \acs{QR} technologies are used in a relationship depends on the person itself.
As further concerns, they stated that users use applications on devices that are also used for other purposes, e.g. such devices as smart phones, and that these devices are connected to the Internet. 
They concluded their argument that it is not a single process of tracking the data, which leads to problems. Rather, the problem lies in the fact that third parties collect the data on the devices that track the data, and so they get the data with existing network connection.

A solution for this concern could be applications on devices which does not communicate with third parties servers via a network connection. This could be devices that only track the data the user want to quantify, without network connection, which lead to use an device in addition to the smartphone.
Or, go a step backwards, the user can track that data completely without digital technologies, e.g. as mentioned above with the sympto-thermal method model by Roetzer.
But the tracking without digital devices is difficult to realize if there is a possibility to support technology tracking. Digital helpers make sense, they are usually reliable and make everyday life easier. Thus, manual tracking is not an alternative.

Users of \acs{QR} technologies, tracking devices, smartphone, social network and others should always be aware of the amount of data which are disclosed and also to whom. Providers and third parties are commercial interested in these data. In addition, the privacy and consent declarations should be noticed and/or simplified, as Anaya et al. in \cite{anaya2018ethical} suggested.

User should also aware whether their data are used for big data science, e.g. with the tracked data in the Glow application.

%TODO: Es fehlt ein UNDO für Datenteilung

Even the usual methods for security arrangements are not sufficient in certain situations. So they are suitable for attacks by strangers, but not for people who are very close and so know many intimate details.
\begin{quote}
	We found that the typical vectors of remote account compromises are technical mundane. Frequently, abusers are able to use their knowledge of the victim's personal details to infer passwords or correctly answer their security questions and reset their password [...]. \cite{freed2018stalker}
\end{quote}
