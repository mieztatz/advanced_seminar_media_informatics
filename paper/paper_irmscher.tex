% !TEX spellcheck = en_US
% !TeX program = pdflatex
% !TeX TXS-program:bibliography = txs:///bibtex
% !BIB program = bibtex

%% LMU-MI-HS-Template
%% This template is an adaptation of the IEEE InfoVis/Vis format
%% http://www.cs.sfu.ca/~vis/Tasks/camera_tvcg.html
%% Last update: Bastian Pfleging, 05.2016

\documentclass[journal]{vgtc}                % final (journal style)
\usepackage[english]{babel}
\usepackage{mathptmx}
\usepackage{graphicx}
\usepackage{times}
\usepackage[hyphens]{url}
\usepackage{float}
\usepackage[hidelinks]{hyperref}
\usepackage[nolist]{acronym}

\usepackage[backend=bibtex, style=numeric, isbn=true, doi=true, maxnames=99]{biblatex}
\addbibresource{literature.bib}

\DeclareGraphicsExtensions{.pdf,.jpg,.pdf,.mps,.png}
\graphicspath{{img/}} 

\usepackage[official]{eurosym}

\normalfont

\begin{acronym}[Bash]
	\acro{QR}{Quantified Relationship}
\end{acronym}

%% Paper title.
\title{Intimate data in relationships: Tracking, sharing, surveillance - personal boundaries?}

%% Put your name here
\author{Diana Irmscher}
\authorfooter{
\item
  Diana Irmscher is studying Media Informatics at the University of Munich, Germany, E-mail: d.irmscher@campus.lmu.de
\item
  This research paper was written for the Media Informatics Advanced Seminar 'Advanced Seminar in Media Informatics',
  2018
}


%% Abstract section.
\abstract{
	% One or two sentences providing a basic introduction to the field, comprehensible to a scientist in any discipline.
	Self-tracking has become commonplace in the century of digitalization. People are tracking themselves or rather personal information. Moreover, they are storing and sharing this data. 
	
	% Two to three sentences of more detailed background, comprehensible to scientists in related disciplines.
	The Quantifying Self movement has inspired this trend. Today it is possible to track quantifiable data like heartbeat or sleeping pattern, but also "'not measurable"' data like moods, feelings and behaviors, which have a high intrinsic value. Such tracked data can be highly intimate, e.g. data of sexual and reproductive activities or intimate relationship. 
	
	% One sentence clearly stating the general problem being addressed by this particular study.
	In this work, it is considered why people tracking themselves, and storing and also sharing such intimate data.
	
	% One sentence summarizing the main result (with the words “here we show” or their equivalent).
	
	% Two or three sentences explaining what the main result reveals in direct comparison to what was thought to be the case previously, or how the main result adds to previous knowledge.
	
	% One or two sentences to put the results into a more general context.
	
	% Two or three sentences to provide a broader perspective, readily comprehensible to a scientist in any discipline, may be included in the first paragraph if the editor considers that the accessibility of the paper is significantly enhanced by their inclusion. Under these circumstances, the length of the paragraph can be up to 300 words. (This example is 190 words without the final section, and 250 words with it).
}

%% Keywords that describe your work.
\keywords{self-tracking, self-quantification, personal informatics, intimate data, data ethics, privacy}

%%%%%%%%%%%%%%%%%%%%%%%%%%%%%%%%%%%%%%%%%%%%%%%%%%%%%%%%%%%%%%%%
%%%%%%%%%%%%%%%%%%%%%% START OF THE PAPER %%%%%%%%%%%%%%%%%%%%%%
%%%%%%%%%%%%%%%%%%%%%%%%%%%%%%%%%%%%%%%%%%%%%%%%%%%%%%%%%%%%%%%%%

\begin{document}

\firstsection{Introduction}

\maketitle

% ------------------------------------------------------------------------------
%
% This is only an exemplary structure for your paper! Feel free to change the names of the sections
% and subsections! 
%
% ------------------------------------------------------------------------------
% Move 1: Establishing a Territory

% Step 1: Claiming importance and/or
% Step 2: Making topic generalization(s) and/or
% Step 3: Reviewing items of previous research
In the century of digitalization there are many opportunities offered to perceive the self and own life in a different way as before. Tracking and quantifying is commonly used. Nowadays many people are engaged in tracking their data. They are tracking and also sharing this information with other people, like friends or like-minded people. 

% Move 2: Establishing a Niche

% Step 1A: Counter-claiming or
% Step 1B: Indicating a gap or
% Step 1C: Question-raising or
% Step 1D: Continuing a tradition

But there are many different types of data, which can be tracked. Such data like heartbeat or sleeping pattern does not seem to be too intimate when tracking and sharing, but how about data in intimate relationships and sexual behaviors? 

% Move 3: Occupying the Niche

% Step 1A: Outlining purposes or
% Step 1B: Announcing present research
In this work, the collecting, tracking, storing and sharing of intimate data in romantic relationships is investigated.
Therefore, the following questions will be answered by searching for  literature and studies in this scientific field:
 \begin{enumerate}
 	\item What data is perceived as intimate? In what circumstances?
 	\item Why do people track intimate data in relationships?
 	\item What do they do with, e.g. tracking, storing, sharing and discussing and with whom?
 	\begin{enumerate}
 		\item Do they over-trust the tracked data?
 		\item How do they perceive their tracked data?
 	\end{enumerate}
 \end{enumerate}
For answering of the mentioned above questions a research of literature and studies on collecting and tracking intimate data in romantic relationships is carried out. The answering of the first question is not as easy as it seems. Therefore, several definitions from different source are collected.

% Step 2: Announcing principal findings
Intimate data are searched, tracked, stored and shared in every kind of relationship, from the beginning until to termination.
The most different types of intimate data are collected from Facebook and Tinder and tracked by technologies for sexual activities and measuring a woman's cycle.
% Step 3: Indicating research article structure

In section \ref{sec:terms_of_definition} the term \textit{intimate} is defined by gathering different definitions related \acl{QR}-technologies and intimate surveillance.
In section \ref{sec:life_course} the so called life course of intimate data is defined including four conditions in which an intimate relationship could be.
The following section \ref{sec:consideration_life_course_conditions} describes the conditions with regarding to the intimate data that are searched for, tracked, stored and shared in these circumstances.
In section \ref{sec:risks} the risks related to the use of such technologies in a romantic relationship is investigated.
Section \ref{sec:conculsion} includes a summarization of the work with a short view for future steps.

%\section{Related Work}
%\label{sec:relatedWork}
%\cite{doi:10.1080/13691058.2014.920528} shows an critical analysis of a special type of digital health care devices, such as technologies for tracking and sharing users sexual and reproductive activities and functions. Some of these smart-phone apps available in the Apple App Store and also the Google Play Store are investigated of their sociocultural, ethical and political implications. It is shown that such applications emerging ethical and privacy implications, e.g. perpetuate nominative stereotypes and assumptions. It suggests a queering of such technologies and their use.
%
%In \cite{doi:10.1080/15265161.2017.1409823} the research is focused on \ac{QR}. In this work a detailed ethical analysis is provided. The authors found eight core objections to \acs{QR} and investigated these critical. They found out that despite criticism the \acs{QR} tracking technologies can be rated as helpful to support intimate relationships.
%
%In \cite{sjoklint2015complexities} the interplay of technology, data and the self-experience while using self-tracking technology is been investigated. In this work, technology for tracking movement and sleeping activities was used. The investigation shows, that using self tracking devices are not necessary to change behaviors. It is more useful as a re-focusing device. It also shows that the user experiences tends between rational and emotional behaviors when reflecting the tracked data.
%
%In \cite{choe2014understanding} 52 video interviews were taken to understand users, which tend to use self-tracking technologies more than other people do, like share best practices and mistakes through talks, blogging and conferences. The topic of the interview includes, how the users did track themselves and what did they learn. Furthermore, in this work several common pitfalls to self-tracking were found. At the end it is suggested for future research on these pitfalls.

\section{Definition of Terms}
\label{sec:terms_of_definition}
In this section the term \textit{intimate} is defined. It is considered which data is perceived as intimate and in which circumstances. There is no agreed-upon definition to find. Hence, several definitions from different source are collected.

The perceiving of what is intimate depends on several factors.
In general it has to be differentiated between the culture, how a human is perceiving the self and which factors are shaping the sociocultural live \cite{carrithers1985category}. It is not possible to consider all well-known cultures in this work, therefore the focus is limited to the scrutiny of the western civilization. 
\begin{figure}[htb]
	\centering
	\includegraphics[width=\linewidth]{img/cluster_heart.png}
	\caption{Visualization of possible intimate data, which are arising from using such digital technologies.}
	\label{fig:cluster}
\end{figure}
In the western civilization or rather in the \ac{EU} privacy and data security gaining more attention since the \ac{GDPR} is applied \cite{albrecht2016gdpr}.
However, the state of a person in the society is defining the personal perception of privacy and data security, and the personal view as well. What is perceived as intimate depends on these factors.
But these can not be defined in a few sentences, the topic is too complex and not measurable. Furthermore, it is subjective. For the individual, the perception of whether data are intimate or not is different. 
Several works are focused on intimate data in different contexts. However, which data is intimate or what people perceive as intimate is not clearly defined. Due to this, some descriptions are summarized to give a rough outline.

The focus in Danaher et al. \cite{doi:10.1080/15265161.2017.1409823} is on intimate interpersonal relationships. In this work no clear definition of the term intimate data is presented. They argued that it does not need a precise definition to get an understanding of intimate relationships. To describe a romantic relationship the authors wrote:

\begin{quote}
	[...] we trust that most readers' intuitive sense of those terms [..] will be adequate for our arguments to make sense. That said, "romantic relationship" might usefully be thought of as a cluster concept, with paradigmatic examples in the middle, and less paradigmatic examples clustered around it, each one different along various dimensions (e.g., the degree to which sexual interaction is central to the relationship).
\end{quote}

If it is possible to define an intimate or romantic relationship in such a way, this concept will also work for the term \textit{intimate}. 
It can be visualized in a cluster of different types of intimate data, which are assigned to corresponding activities, e.g. fertility tracking.
In figure \ref{fig:cluster} several topics related to the term intimate data are collected and brought in relation to each other. At this point it must be emphasized that this does not cover the complete field, in which intimate data would be collected, tracked, shared and monitored. Rather it is an summarization of terms and descriptions which come up in this work.

%from Levy \cite{levy2014intimate}, Danaher et al. \cite{doi:10.1080/15265161.2017.1409823}, Lupton \cite{doi:10.1080/13691058.2014.920528} and more. 

%The idea to use an cluster concept can be thought of one step further. The sensitivity or level of intimate data could be arranged in some sort of data hierarchy. Form IT-Security Management it is known to evaluate risks by assigning a probability and to classify accordingly (see documentation of Federal Office for Information Security (BSI) \cite{bsi}). In this table I want to classify the data summarized above based on their sensibility.
%%TODO: Tabelle über die oben genannten Daten einfügen mit Einstufung; Stufung bestimmen

To give another understanding of what is meant with the term \textit{intimate} it is quoted from the work by Lupton \cite{doi:10.1080/13691058.2014.920528}, which describes an application for smart phones:
\begin{quote}
	The	Glow app brings male partners into the equation by sending them a digital
	message when their partner is in her fertile period and reminding them to bring her flowers	[...]. This app also tracks menstrual and ovulation indicators, as well as asking women to enter details of their sexual encounters, including sexual positions used, whether or not they had an orgasm and whether they experienced emotional or physical discomfort during sex. It employs the aggregated data from other users to refine predictions of ovulation and fertility for the individual user. [...]
\end{quote}
This paragraph describes a sort of tracking which is also called \textit{intimate tracking} (defined by \cite{doi:10.1080/15265161.2017.1409823}).

We can find intimate data also in other contexts, e.g. as mentioned above in health care.
In general it can be said that the perception of what is perceived as intimate depends on the context of prevailing situations. The health of an employee may be an intimate information, e.g. whether an employee is pregnant. The fact that insurance companies offered their customers a discount if they disclosure their Facebook profile shows how relevant such types of data are for different purposes. \footnote{\url{https://www.heise.de/newsticker/meldung/Versicherung-wollte-Hoehe-der-Kfz-Versicherung-aus-Facebook-Posts-errechnen-3454410.html}}
%TODO: finde Quellen zu Aufsätzen intimer Daten in Verbindung mit Drogensucht
But the focus in this paper is limited on intimate data in relationships, therefore it is referred to the figure \ref{fig:cluster} above. This should give a overall understanding for the following sections. 


\section{Life course of intimate relationships}

Levy \cite{levy2014intimate} has defined a so called \textit{life course of intimate relationships}. This course includes four conditions of romantic relationships (see Figure \ref{fig:live_course}).
\begin{figure}[htb]
    \centering
	\includegraphics[width=\linewidth]{img/life_course_of_intimate_surveillance.png}
	\caption{The life course of intimate surveillance}
	\label{fig:live_course}
\end{figure}

In each of these conditions (potential) partners can use technologies for different purposes.

\textit{TODO: Introducing and explaining of conditions...}


\section{Consideration of each condition in life course}
\label{sec:consideration_life_course_conditions}

\textit{TODO: For each state, the results from the individual papers are collected.
Further individual distinctions can be derived from this (for instance in B it is is to be distinguished between intimate tracking and intimate gamification).
The types of possible data collection, tracking and also sharing is to be reported for each condition in life course.}

\subsection{Condition A: Dating - Scoping out potential intimates}
\label{subsec:A}
At the beginning of a potential relationship you want to know more about the person person of our interest. Due to this, you collect data about this person. 
\subsubsection{Searching for information}
A good way the get relevant information is using a standard social network like Facebook \footnote{\url{www.facebook.com}} or using Google search. Monitoring a person on Facebook is known as Facebook stalking \cite{levy2014intimate}. To stalk another person on Facebook undiscovered, much articles has been written about \cite{sueddeutsche_fb_stalking}. With the Website stalkscan.com \footnote{\url{https://stalkscan.com}} it is possible to get all public entries from a persons Facebook profile site which is public by only one mouse click. Surley, it can only shows what is already set public, still it make it more easily to stalk another person very quickly.
Within this website as tool is also avoided to give an involuntary like by clicking through the photographs, for instance.
The Google search mentioned at first is known as \textit{google someone}. With this method is it possible to get information from every source which is findable for the search engine \cite{nolan2005hacking}. Also for this topic there are many article how to \textit{google someone}. For instance, the search on images is of high interest \footnote{\url{https://www.lifewire.com/google-people-search-3482686}}.

\begin{table*}
	%% Table captions on top in journal version
	\caption{Interrelated types of data in \acl{QR}, source from \cite{doi:10.1080/15265161.2017.1409823}}
	\label{tab:typ_of_QR}
	\scriptsize
	\begin{center}
		\begin{tabular}{|p{4cm}|p{11cm}|p{2cm}|}
			\hline
			Intimate tracking &  Collection of all (measurable) data that can arise through intimate behaviors (in a relationship), e.g. number of partners, number of sexual encounters, duration of sexual encounter, or romantic behaviors (gifts, help in the household, attention) & SexTracker \newline SexKeeper \newline Nipple \newline Lovely \newline kGoal \\
			\hline
			Intimate gamification & Use of gamelike incentives to change or improve the behavior in a romantic relationship; Playful learning to lead a successful relationship & - \\
			\hline
			Intimate surveillance & Use of technologies to monitor intimate partners & - \\
			\hline
		\end{tabular}
	\end{center}
\end{table*}
\subsubsection{Creating and providing information}
The topic in this condition A is not only searching for data about someone, but also create such data. Levy \cite{levy2014intimate} mentioned the application Lulu as a tool to create data for use in prospective relationships. The focus of this application is on campus life. The app Lulu gives young women the opportunity to review male students and friends, with which they are connected on Facebook. The review contains information in relation to humor, manners, look and style, sex and kissing. The review giving by the female users is anonymously. In the first version, each male fiend on Facebook could be reviewed in this app. But after concerns related to privacy of reviewed male Facebook users, a review can only be committed for such male user which have explicitly allowed to this.

Furthermore, such services that combine online dating with user's geographical location are well known. Tinder is a widespread location-based dating service. The app shows potential people with different interests (e.g. romantic relationship) near to the user's location or next holiday destination \footnote{\url{https://tinder.com}}. By showing the user several profiles he/she can deside to swip right for a like. If the other person does also a right swip, it is a match. Now the user can exchange messages, for instance to get a date. The princip sound easy, but isn't at all. By using these app, a huge among of intimate data is collected. 
First of all, the tinder app is connected to Facebook and Instagram, a photo-sharing social networking service, owned by Facebook itself. In order to this there is a huge commercial interest to assume. Judith Duportail demanded access to here personal data under European data protection law after four year using the tinder app. The respons was an over 800 site report containing diffrent types of data like Facebook likes, information about education, age-rank of men she was interested in, number of Facebook friends, when and were every online conversation with her matches happened, also interests and jobs, pictures, sexual preferences. The list contains a huge amount of intimate data. In her article Duportail writes, she was amazed by how much information she was voluntarily disclosing. This was also called secondary implicit disclosed information. Firms have an increasing interest in gathering personal data from user's activities \cite{taylor2009privacy}. This results in a trade-off for the user - use the system and accept privacy concerns due to the commercial interest from the provider, or abstain the service.
Nevertheless all concerns, users reveal their data very quickly, as shown in Tait et al. \cite{tait2015hello}. 
Users who tend to gain confidence quickly, therefore, also more quickly reveal more information. In addition, this study showed that higher profile activity increases the amount of information desired. 
That means, users who maintain an active profile and present activity also receive more and higher information from other user's rather than users of profiles that provide barely information. The disclosure of information is determined in part by the personality of the user and the context in general. This affects how users surround their data online and with strangers. They found out that in only 6 - 10 minutes a user can extract the full name and date of birth from a conversation. Within these information it is easy to get further data about the person via Google search and Facebook, for instance.

In Nandwani et al. \cite{10.1007/978-3-319-61542-4_32} it was examined how quickly users reported their data to strangers and, above all, which data. For the study, an automatism was developed to contact 100 Tinder users. The study was a single blind study, so users did not know at the moment that they were writing with a Chat-bot. The evaluation of the data yielded the following results: Most of the published data was personal data, for instance: full name, date of birth, phone numbers, work details, email-addresses, complete address and other data that will not be listed here.

Why are this data disclosed to strangers in online platforms and apps? As mentioned above, the user trusts in the authenticity of the other within an active profile account. Also they do not reflect the impacts of disclosure the personal and also intimate data. For this purpose, Nandwani et al. \cite{10.1007/978-3-319-61542-4_32} suggest an virtual assistant in such applications like Tinder, which study the relationship between the users by parameters and inform the user which information should be reveal in the conversation.

\subsection{B: Tracking intimate and romantic practices}
\label{subsec:B}
The potential of creating, collecting and tracking intimate data rises if the romantic relationship between two individual goes deeper. A romantic relationship in which intimate data were tracked is named a \ac{QR}.
Danaher et al. \cite{doi:10.1080/15265161.2017.1409823} describing in their work three categories of intimate data which can be tracked in a \acs{QR}. In table \ref{tab:typ_of_QR} the three categories are summarized with a descriptions and examples.

In the following the categories intimate tracking and intimate gamification are considered in more detail.
The third category intimate surveillance will be discussed in section \ref{subsec:D}.

\subsubsection{Intimate tracking}
For the tracking of intimate data, there are a variety of apps that can be used for it.
The apps usually track the following data about sex life: \cite{doi:10.1080/15265161.2017.1409823}: 
\begin{itemize}
	\item number of partners
	\item number of "sessions" per partner
	\item sexual positions used during theses sessions
	\item number of thrusts per session
	\item duration of these sessions
	\item number of calories burned per session 
\end{itemize}
This list only mentions the most common. There are many more variants of intimate data that can be tracked. As Kelly \cite{kelly2017inevitable} mentions, nearly everything is tracked that is possible. That may not cover the big crowd, but that's also practiced.

The data is voluntarily and automatically tracked using such technology \cite{doi:10.1080/15265161.2017.1409823}. That means, the data is either actively provided by users or automatically recorded, e.g. by running the app in the background and recording audio recordings.
This type of tracking or communication is also referred to as \textit{participatory surveillance}. As Lupton \cite{doi:10.1080/13691058.2014.920528} writes, this includes looking at oneself, but for one's own purpose. Self-tracking is often associated with self-reflection, but it has less to do with it \cite{lupton2016quantified}. Rather, it is a visualization and reflection of the collected numbers. But the reflection of the self in this context involves much more than the visualization of the numerical data. This is more of a strict focus on the pure numbers. These numbers are only objectively perceived, and no longer associated with the subjective activity or context to which they once belonged.
Often, these apps also contain elements for the gamification of the mission or goals.

\subsubsection{Intimate gamification}
Another observation is the gamification in this area of tracking. Users are encouraged to quantify their sex lives in order to measure their performance and compare themselves with other users \cite{doi:10.1080/13691058.2014.920528}. This type of quantification mainly focuses on the male user.

One consequence of using such technologies may be the reinforcement of gender stereotypes \cite{doi:10.1080/13691058.2014.920528}. The algorithm defines the goals that users use to orient and measure themselves against. The individuality is lost.

In addition, this type of feedback does not necessarily have to be of good quality or have a lasting effect on relationship life. \cite{doi:10.1080/15265161.2017.1409823}. After all, a good relationship is not measured by how much sex you have or how long it lasts. As explained in the section \ref{sec:terms_of_definition} above, there are many more components that make a good relationship.

\subsubsection{Objections}
The automatic recording of such data in an app can be very questionable, because the danger is great that the user is not aware of it. Most users do not read the small print conditions of these services before using them \cite{levy2014intimate}.

Also, the sole quantification of a relationship does not necessarily lead to an improvement of the relationship skills. Rather, these types of behavioral change supports gender stereotypical reinforcement. That would be a very retrograde development compared to the current perception of our conception of love and sexuality. %TODO: Quelle für diese Behauptung suchen
In addition, as already mentioned above, the users perceive the data objectively only by quantifying the activities, similar to a sport activity like running. The reflection of the real activity is lost.

Users share this data with like-minded users, keep it for themselves and do not share it, or share it with their intimate partner.
The mere possibility of sharing this data brings with it a significantly larger audience \cite{doi:10.1080/13691058.2014.920528}. This also influences the willingness to disclose intimate data to strangers.
Users also share the data for the purpose of comparison with other users. The gamification which is often used in such apps also supports this in addition.

\subsection{C: Monitoring Fertility}
This section will focus on tracking the cycle and fertility of female users. These types of data are therefore very intimate. So far, they have been collected only in conjunction with a medical treatment and evaluated with the doctor.

The cycle and thus the connected fertility of the woman has been "monitored" for a long time. The exact beginning is unknown, it has been writing about it since the 1920s in scientific medical \cite{rotzer1988geschichte}. %TODO: Prüfen
With Josef Roetzer the symptothermal method became well known \cite{roetzer1968erweiterte}. With this method the cycle could be monitored and the fertile days could be determined exactly with a few differences.

In the age of digitization, there are of course digital technologies that support the female user to monitor the data.

In der folgenden Tabelle sind einige (bekannte) App aufgelistet.

\begin{table}
	%% Table captions on top in journal version
	\caption{App for tracking the cycle}
	\label{tab:typ_of_QR}
	\scriptsize
	\begin{center}
		\begin{tabular}{|p{1.5cm}|p{1cm}|p{4.5cm}|}
			\hline
			Bezeichnung & BS & Beschschreibung \\
			\hline
			\hline
			myNFP &  iOS \newline Android &  myNFP wertet den Zyklus nach der symptothermalen Methode (NFP) aus. Alle wichtigen Parameter für die Auswertung werden von der Nutzerin selbst eingetragen.
			\\
			\hline
			Kindara Fertility \& Ovulation  & iOS \newline Android & is based on the Fertility Awareness Method; supports the Sympto-Thermal method and can be used for Natural Family Planning (NFP) \\
			\hline
			Lily & iOS & es wird regelkonform nach symptothermaler Methode ausgewertet oder anhand von Durchschnittswerten anderer User; möchte man die App im vollen Funktionsumfang nutzen, wird ein Beitrag erhoben; dafür garantiert der Hersteller, dass die Daten nicht durch Dritte ausgewertet werden, dass die personenebzogenen Informationen nicht gespeichert werden udn auf keinem Server ein Backup der personenbezogenen Daten liegen \footnote{\url{http://whimsicallily.com/lily/en/privacypolicy.php}} \\
			\hline
			Glow & iOS \newline Android & \\
			\hline
		\end{tabular}
	\end{center}
\end{table}
Laut dem Hersteller myNFP werden die sensibeln Daten nicht durch Dritte verarbeitet. Weiterhin werden sowenig Daten erfasst wie möglich. Die Daten sind anonym und lassen nicht auf die Person zurückschließen. Der Herstellter begründet das mit dem Agrument, dass die App pro Monat  2,50 \euro{} \footnote{\url{https://www.mynfp.de/datenschutz}}.

ZUr Kindara App finden sich ebenfalls Angaben zum Datenschutz, diese sehen allerdings anders aus als in der vorhergehenden App \footnote{\url{https://www.kindara.com/privacy-policy}}:
\begin{quote}
	Kindara collects and uses the information you provide to us when you use the Kindara Service. Information that Kindara may collect includes: name, date of birth, e-mail address, fertility-related data and other family planning and health-related information you provide. You may consider some of this information to be sensitive so you should choose carefully regarding whether and if you will use the Service.
\end{quote}
 Da die App kostenlos angeboten wird, liegt die Vermutung nahe, dass die Daten weiter verarbeitet werden. ZUr App wird ein zusätzliches Device angeboten, mit welchem die Aufwachtemperator gemessen werden kann. Das Gerät connecet sich mit der App automatisch, wenn die Temperatur gemessen wird. Die Daten werden par Bluethoot an die App übermittelt \footnote{\url{https://www.kindara.com/wink}}.

\subsection{The app Glow}
Auf die Glow app möchte ich gesondert eingehen, da sehr viele Arbeiten darüber schreiben \cite{doi:10.1080/15265161.2017.1409823}, \cite{levy2014intimate} und \cite{doi:10.1080/13691058.2014.920528}.
\begin{figure}[htb]
	\centering
	\includegraphics[width=\linewidth]{img/Glow-App-review-screenshot-1.jpg}
	\caption{Die Glow App erfasst ein Vielzahl an intimen Daten (Bildquelle: \cite{glowApp})}
	\label{fig:glow_app}
\end{figure}
Die App wurde vom PayPal founder Max Levchin in 2013 gestartet, und bietet genüber den vielen anderen Apps im Bereich Fruchtbarkeit und natürliche Verhütung eine große Konkruenz. Die Glow App trackt eine Vielzahl an Daten, unter anderen die Menstruation, position and firmness of a womens's cervix, sexual intercourse (with the women's position during ejaculation), "wheter they had an orgasm [and] wheter they experienced emotional or physical discomfort during sex" \cite{doi:10.1080/13691058.2014.920528} . Außerdem kann die Stimmung der Nutzerin getrackt werden.
Der Unterschied zu anderen Apps ist, die Glow app macht die Sammlung intimer Daten zu einer Familienangelegenheit. Die Partner der Nutzerin werden dazu eingeladen einen mirror app herunterzuladen, und zusätzliche Daten anzugeben \cite{levy2014intimate} Die App sendet dem Partner außerdem Nachrichten über den akteullen Stand der Period der Partnerin und erinnert an eine Aufmerksamkeit, wie z.B. Blumen oder eine nette Nachricht.

Die Daten der Nutzer werden gesammlet ausgewertet, um aus der großen sammlung bessere Vorhersagen für die individuellen Nutzerin angeben zu können.

Danaher et al. \cite{doi:10.1080/15265161.2017.1409823} argumentieren unter dem Punkt Gender Relationship Objection, dass diese Art der Technologien Frauen zu einem Objekt der Überwachung und Quantifizierung machen. Diese Technologien geben den Anschein, also sei der Zyklus einer Frau unüberwacht chaotisch und nur mit strenger Kontrolle "regulierbar".
ZUdem würden diese App die Entwicklung und Verstärkung von geschlechtsspezifischen Stereotypen fördern.

Die Angabe solcher intimer Daten ist zum teil sehr bedenklich, wenn die Nutzerin nicht beachtet, wie die Daten weiter ausgewertet werden. Sicherlich können diese Technologien hilfreich sein in der Auswertung der erhobenen Daten, und erinnern an die tägliche Messung. Leider werden diese sehr sensiblen Daten auch für commerziellen Zwecke genutzt. 



The first three conditions can be summarized well in the following graphic.
\begin{figure}[htb]
	\centering
	\includegraphics[width=\linewidth]{img/d372d-intimate2bsurveillance-122.png}
	\caption{Summarizing of conditions A, B and C \cite{ethicsOfSurveillance}}
	\label{fig:intimate_surveillance}
\end{figure}
\subsection{D}
"Now that mobile digital technologies that can be used for surveillance are part of everyday social life. "\cite{doi:10.1080/13691058.2014.920528}.
\label{subsec:D}
\begin{description}
	\item[A] \textit{Data collection at the beginning of a relationship, Facebook stalking, potential partner googling, Tinder. In the following: why is this used or why are these data collected, recorded etc. Subsequently, how do people perceive this, influence of data on perception}
	\item[B] \textit{Categorization in intimate tracking and intimate gamification from \acl{QR}: example of these apps and tracking devices. What added value do they have in the relationship? What's in it? How do people perceive that (Quantifying, over-trust in numbers).}
	\item[C] \textit{Drafting the role of women at this stage of a relationship: many apps and devices for tracking women (cycle, fertility, etc.).}
	\item[D] \textit{Category intimate surveillance from \acl{QR}: main emphasis:
	Tracking the partners in a relationship: acceptable or not by mutual agreement? Does that affect the relationship, or the mutual trust? There is no investigation until now (continue at the end (conclusion, further work)).}
\end{description}

\section{Risks}
Lupton writes in \cite{doi:10.1080/13691058.2014.920528}:
\begin{quote}
	"Now that mobile digital technologies that can be used for surveillance are part of everyday social life.
\end{quote}
Since the technologies discussed above are in daily use, they pose some risks to the users privacy, the perception of themselves and also of the relationship which they lead.
In this section these risks are summarized to give an overview.
The overview is divided into the three categories quantification, trust in a relationship and user privacy.

\subsubsection{Quantification: Perception and rating of the self and the relationship}
Due to the various ways in which intimate data can be tracked, there is a risk of losing the actual reference to the data \cite{doi:10.1080/13691058.2014.920528} and \cite{lupton2016quantified}. In condition \ref{subsec:B} it was mentioned that by tracking of sexual activities the actual act later is only perceived by numbers, thus the act is quantified. The quality or actual perception by the user can be lost. Or put another way, the user is lost in a jumble of numbers \cite{kelly2017inevitable}.
When using these technologies, the user should be aware of why he or she is using them and what these data are actually collected for \cite{doi:10.1080/15265161.2017.1409823}. 
Often it is the case that many users are interested in tracking at the beginning, but after a while they give up using the tracking device and are no long interested in \cite{sjoklint2015complexities}.

\subsubsection{Trust: unknowingly and knowingly tracking by intimate partner; over-trust in data only}
The surveillance of the partner without his consent is on the topic of \acs{QR}-Technologien out of the discussion, as Danaher et al. in \cite{doi:10.1080/15265161.2017.1409823} argue. This is clearly the abuse of the data. This includes also the use of such apps as Flexispy \footnote{\url{https://www.flexispy.com/en/}}.

However, the approach that partners voluntarily monitor each other as described in D could also create problems related to the use of such an app or tracking device. This includes for instance the abuse by a dominant partner that might force the use of such software in the relationship. That would not be mutual agreement.

\subsubsection{Users privacy risks}
In Danaher et al. \cite{doi:10.1080/15265161.2017.1422294} the risks associated with the use of \acs{QR}-technologies are summarized.
The authors argumented that the concerns " [...] of the privacy-invading elephant lurking in the room [...]" are not alone a problem of a single person, but also involves one or more persons. However, this is exclusively private and an interpersonal matter, for instance if ever and with what device \acs{QR} technologies are used.
As further concerns, they stated that users use apps on devices that are also used for other purposes, such as smart phones, and that these devices are connected to the Internet. 
They conclude their argument that it is not a single process of tracking the data, which leads to problems. Rather, the problem lies in the fact that third parties collect the data on the devices that track the data, and so they get the data within existing network connection.
Remedy would create devices that only track without transmitting the data.
It used to be tracked without digital helpers, see  \ref{sec:c} the symptothermal method model from Roetzer.
%\begin{enumerate}
%	\item \textit{Quantification (perception and rating of the self and the relationship)}
%	\item \textit{Trust (unknowingly tracking by intimate partner, over-trust in data only)}
%	\item \textit{Privacy (risks, current news, data gaps, etc.)}
%\end{enumerate}

\section{Conclusion and future work}
In this work, different conditions in a relationship  were considered in which intimate data can be searched, collected tracked and shared.
Further, an attempt was made to formulate a definition for the term \textit{intimate data}. It also summarized how people perceive this data and how the data affects their perception.

For further work it is to investigate how partners in a well-functioning relationship perceive a mutual surveillance.

\label{sec:conculsion}
\section{Achievements}

\printbibliography
\end{document}
