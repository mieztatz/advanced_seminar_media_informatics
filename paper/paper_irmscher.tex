% !TEX spellcheck = en_US
% !TeX program = pdflatex
% !TeX TXS-program:bibliography = txs:///bibtex
% !BIB program = bibtex

%% LMU-MI-HS-Template
%% This template is an adaptation of the IEEE InfoVis/Vis format
%% http://www.cs.sfu.ca/~vis/Tasks/camera_tvcg.html
%% Last update: Bastian Pfleging, 05.2016

\documentclass[journal]{vgtc}                % final (journal style)
\usepackage[english]{babel}
\usepackage{mathptmx}
\usepackage{graphicx}
\usepackage{times}
\usepackage[hyphens]{url}
\usepackage{float}

\usepackage[backend=bibtex, style=numeric, isbn=true, doi=true, maxnames=99]{biblatex}
\addbibresource{literature.bib}

\DeclareGraphicsExtensions{.pdf,.jpg,.pdf,.mps,.png}
\graphicspath{{img/}} 

\normalfont

%% Paper title.

\title{Intimate data in Personal Informatics: Tracking, sharing and personal boundaries?}

%% Put your name here
\author{Diana Irmscher}
\authorfooter{
%% change name, course (Media Informatics/Informatics/etc) and email for the footer
\item
  Diana Irmscher is studying Media Informatics at the University of Munich, Germany, E-mail: d.irmscher@campus.lmu.de
\item
  This research paper was written for the Media Informatics Advanced Seminar 'Advanced Seminar in Media Informatics',
  2018
}


%% Abstract section.
\abstract{
Sum up your work and the ideas behind it in 150 to 250 words.
} % end of abstract

%% Keywords that describe your work.
\keywords{Fill, In, Your, Own, Keywords}

%%%%%%%%%%%%%%%%%%%%%%%%%%%%%%%%%%%%%%%%%%%%%%%%%%%%%%%%%%%%%%%%
%%%%%%%%%%%%%%%%%%%%%% START OF THE PAPER %%%%%%%%%%%%%%%%%%%%%%
%%%%%%%%%%%%%%%%%%%%%%%%%%%%%%%%%%%%%%%%%%%%%%%%%%%%%%%%%%%%%%%%%

\begin{document}

\firstsection{Problem Statement}

\maketitle

% ------------------------------------------------------------------------------
%
% This is only an exemplary structure for your paper! Feel free to change the names of the sections
% and subsections! 
%
% ------------------------------------------------------------------------------
Nowadays many people are engaging in self-tracking their body data. They are tracking the data, and also they are sharing these with other people, like friends or like-minded people. There are so many different types of data, which we can track. In this work, the intimate data of our body or rather of our life will be considered. Therefore, the type of intimate data should be defined. What data is perceived as intimate and in what circumstances?
Why do people do track intimate data?
What do they do with it? Tracking, storing, sharing (with whom?) and discussing (with whom)? Do they overtrust this data? How do you perceive these?

Im Zeitalter der Digitalisierung stehen viele neue Möglichkeiten zur Verfügung, sich selbst und das Leben wahrzunehmen. Self-tracking und Quantifiying wird heute von vielen in unterschiedlichsten Formen genutzt. Dafür werden digitale "'Helfer"' genutzt, die das möglich machen. Es gibt viele Arbeiten und Veröffentlichungen über das Thema, diese stehen allerdings vorwiegend im Fokus des Designs solcher "'Helferlein"' \cite{doi:10.1080/13691058.2014.920528}. Diese Helferlein sammeln zum Teil sehr viele Daten, und einige dieser Daten sind als besonder provate bzw. intim einzuordnen. Aber wie nehmen die Personen überhaupt intime Daten wahr? Und unter welchen Umständen. Des weitern gilt zu klären, warum die Leute überhaupt ihre Daten aufzeichnen, und was sie mit diesen Daten machen.
What data is perceived as intimate? In what circumstances?
Why do people track intimate data?
What do they do with it? Tracking, storing, sharing (with
whom?) and discussing (with whom?)


\section{Introduction}
Intimate data are pervieced very different form people. First of all, it is an cultural aspect, and depends on what society think about.


\section{Related Work}

\subsection{What data is perceived as intimate? In what circumstances?}
This question can not be answered so easily. The perceiving what is intimate depends on several factors.
In general it has to be differentiated in the culture, how a human is perceiving the self and what is shaping the sociocultural live \cite{carrithers1985category}. It is not possible to consider all well-known cultures in this work, therefore the focus is limited on the scrutiny of the western world or rather civilization. 

In the western civilization the privacy takes up a lot of space. Nevertheless, the state of a person in the society is defining the personal perceiving of privacy and intimate data. And the personal view, as well.
These things can't be defined in a few sentences, the topic is to complex and not measurable. For the individual, intimate data are different.

Due to this, the definition of what is perceived as intimate for people living in the western civilization, will be shown by the following examples.

The dating app \textit{Tinder} is well known. It is used worldwide 


\begin{enumerate} 
	\item Kultur und Stand
	\item Art der Daten
	\item Gender
\end{enumerate}

\subsection{Evaluation}

\section{Research Plan}

\begin{table}[H]
  %% Table captions on top in journal version
  \caption{Research Plan}
  \label{tab:vis_accept}
  \scriptsize
  \begin{center}
    \begin{tabular}{ll}
      Date & Objective \\
    \hline
      16.04.2018 &  First Meeting  with supervisors, prepare proposal \\
      22.04.2018 & Submit final proposal \\
      08.05.2018 & Submission of first paper draft \\
      11.05.2018 & Submission of 60 sec. presentation \\
      29.05 Sub
	\end{tabular}
  \end{center}
\end{table}


\section{Risk Analysis}

\printbibliography
\end{document}
