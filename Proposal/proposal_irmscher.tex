% !TEX spellcheck = en_US
% !TeX program = pdflatex
% !TeX TXS-program:bibliography = txs:///bibtex
% !BIB program = bibtex

%% LMU-MI-HS-Template
%% This template is an adaptation of the IEEE InfoVis/Vis format
%% http://www.cs.sfu.ca/~vis/Tasks/camera_tvcg.html
%% Last update: Bastian Pfleging, 05.2016

\documentclass[journal]{vgtc}                % final (journal style)
\usepackage[english]{babel}
\usepackage{mathptmx}
\usepackage{graphicx}
\usepackage{times}
\usepackage[hidelinks]{hyperref}
\usepackage{float}
\usepackage[nolist]{acronym}

\usepackage[backend=bibtex, style=numeric, isbn=true, doi=true, maxnames=99]{biblatex}
\addbibresource{literature.bib}

\DeclareGraphicsExtensions{.pdf,.jpg,.pdf,.mps,.png}
\graphicspath{{img/}} 

\normalfont

\begin{acronym}[Bash]
	\acro{QR}{Quantified Relationship}
\end{acronym}


%% Paper title.

\title{Intimate data in Personal Informatics: Tracking, sharing and personal boundaries?}

%% Put your name here
\author{Diana Irmscher}
\authorfooter{
%% change name, course (Media Informatics/Informatics/etc) and email for the footer
\item
  Diana Irmscher is studying Media Informatics at the University of Munich, Germany, E-mail: d.irmscher@campus.lmu.de
\item
  This research paper was written for the Media Informatics Advanced Seminar 'Advanced Seminar in Media Informatics',
  2018
}


%% Abstract section.
\abstract{

} % end of abstract

%% Keywords that describe your work.
\keywords{Fill, In, Your, Own, Keywords}

%%%%%%%%%%%%%%%%%%%%%%%%%%%%%%%%%%%%%%%%%%%%%%%%%%%%%%%%%%%%%%%%
%%%%%%%%%%%%%%%%%%%%%% START OF THE PAPER %%%%%%%%%%%%%%%%%%%%%%
%%%%%%%%%%%%%%%%%%%%%%%%%%%%%%%%%%%%%%%%%%%%%%%%%%%%%%%%%%%%%%%%%

\begin{document}

\firstsection{Problem Statement}

\maketitle

% ------------------------------------------------------------------------------
%
% This is only an exemplary structure for your paper! Feel free to change the names of the sections
% and subsections! 
%
% ------------------------------------------------------------------------------
In century of digitalization there are many opportunities offered to perceive the self and own life in a different way as before. Self-tracking and quantifying is commonly used. Nowadays many people are engaging in self-tracking their body data. They are tracking the data, and also sharing these with other people, like friends or like-minded people. But there are many different types of data, which we can track. Such data like heartbeat or sleeping pattern does not seem to be too intimate when tracking and sharing, but how about data in intimate relationships and sexual behaviorism? In this work, the tracking and sharing of intimate data from our body or rather about our life will be considered. Therefore, the type of intimate data should be defined at first. What data is perceived as intimate and in what circumstances? Farther it will be investigate why people does tracking intimate data, and what they do with it. With whom do they sharing and discussing their intimate data? And finally, it will  be discussed if they over-trust this tracked data and how they perceive these.


\section{Introduction}


\section{Related Work}
\cite{doi:10.1080/13691058.2014.920528} shows an critical analysis of a special type of digital health care device, such as self tracking technologies for tracking and sharing users sexual and reproductive activities and functions. Some of these smart phone apps available in the Apple App Store and also the Google play store are investigated of their sociocultural, ethical and political implications. It is show that such apps emerging ethical and privacy implications, and 

In \cite{doi:10.1080/15265161.2017.1409823} the research is focused on \ac{QR}. In this work a detailed ethical analysis is provided. They authors found eight core objections to \acs{QR} and investigated these critical. They found out that despite criticism, the \acs{QR} tracking technologies can be rated as helpful to support intimate relationships.


\subsection{Justification}

\subsection{Evaluation}

\section{Research Plan}

\begin{table}[H]
  %% Table captions on top in journal version
  \caption{Research Plan}
  \label{tab:vis_accept}
  \scriptsize
  \begin{center}
    \begin{tabular}{ll}
      Date & Objective \\
    \hline
      16.04.2018 &  First Meeting  with supervisors, prepare proposal \\
      22.04.2018 & Submit final proposal \\
      08.05.2018 & Submission of first paper draft \\
      11.05.2018 & Submission of 60 sec. presentation \\
      29.05 Sub
	\end{tabular}
  \end{center}
\end{table}


\section{Risk Analysis}

\printbibliography
\end{document}
